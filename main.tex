\documentclass[12pt]{article}
\usepackage[margin=1in]{geometry}
\usepackage{graphicx}
\usepackage{amsmath}
\usepackage{amssymb}
\usepackage{indentfirst}
\usepackage{float}
\usepackage{lmodern}
\usepackage{physics}
\usepackage{gensymb}
\usepackage[backend=biber,style=numeric]{biblatex}
%\usepackage{biblatex}
\newcommand*\diff{\mathop{}\!\mathrm{d}}
\setlength{\parindent}{15pt}

\newcommand{\pc}{\mathcal{P}_{\zeta,c}}
\newcommand{\pmo}{\mathcal{P}_{\zeta,m}}

\addbibresource{refs.bib}

%opening
\begin{document} 
	
	\begin{titlepage} % Suppresses headers and footers on the title page
		
		\centering % Centre everything on the title page
		
		\scshape % Use small caps for all text on the title page
		
		\vspace*{\baselineskip} % White space at the top of the page
	
		%------------------------------------------------
		%	Title
		%------------------------------------------------
		
		\rule{\textwidth}{1.6pt}\vspace*{-\baselineskip}\vspace*{2pt} % Thick horizontal rule
		\rule{\textwidth}{0.4pt} % Thin horizontal rule
		
		\vspace{0.75\baselineskip} % Whitespace above the title
		
		{\LARGE Perturbations from a Inflationary Spectator Field } % Title
		
		\vspace{0.75\baselineskip} % Whitespace below the title
		
		\rule{\textwidth}{0.4pt}\vspace*{-\baselineskip}\vspace{3.2pt} % Thin horizontal rule
		\rule{\textwidth}{1.6pt} % Thick horizontal rule
		
		\vspace{2\baselineskip} % Whitespace after the title block
		
		%------------------------------------------------
		%	Subtitle
		%------------------------------------------------
		
		
		\vspace*{3\baselineskip} % Whitespace under the subtitle
		
		%------------------------------------------------
		%	Editor(s)
		%------------------------------------------------
		
		
		\vspace{0.5\baselineskip} % Whitespace before the editors
		
		{\scshape\Large Nick Konz\\} % Editor list
		
		\vspace{0.75\baselineskip}
		
		Presentation Partners: Autumn Ficker and Roark Habegger
		
		\vspace{0.75\baselineskip}
% Whitespace below the editor list
		
		\textit{The University of North Carolina \\ Chapel Hill} % Editor affiliation
		
		\vspace{0.75\baselineskip}
		
		\textbf{Honor Pledge:}
		\textit{All work presented here is my own.}
		
		\vspace{3\baselineskip}
			
		\textbf{\_\_\_\_\_\_\_\_\_\_\_\_\_\_\_\_\_\_\_\_\_\_\_\_\_\_\_\_\_\_\_\_\_\_\_\_\_\_\_\_\_\_\_\_\_\_\_\_\_\_\_\_\_\_\_\_\_}
        
\vspace{2\baselineskip}

		
		\vfill % Whitespace between editor names and publisher logo
		
		%------------------------------------------------
		%	Publisher
		%------------------------------------------------
		
		
		\vspace{0.3\baselineskip} % Whitespace under the publisher logo
		
		Due April 26$^{th}$ 2019 % Publication year
		
		{\large ASTR 504} % Publisher
		
	\end{titlepage}

%Any light scalar field that is present during inflation will be perturbed by quantum fluctuations, so it is possible that
%the inflaton is not the primary source of the primordial scalar perturbations. Summarize both the curvaton scenario
%and the modulated reheating scenario and discuss how these models have observational consequences that could
%distinguish them from single-field inflation.
\section*{Abstract}
\par Following the observations of inflationary perturbations in recent years, a number of models for single-field inflation have been all but ruled out \cite{curvaton}. However, if we move from single-field theory to considering the presence of an additional ``spectator'' \textit{curvaton} field during inflation besides the inflaton, a number of inflaton potentials are made viable; this is because with the introduction of an additional field, observed perturbations can be decoupled from the inflaton. I will also consider the presence of a \textit{modulus} spectator field, which generates nonuniform spatial modulations in the decay of the inflaton during reheating \cite{modul}. Both of these spectator fields can also modify the observed perturbation power spectrum, which in turn can modify observerables such as the spectral index, tensor-to-scalar ratio, and the non-Gaussianity in observed fluctuations.
\section{The Primordial Power Spectrum for Multiple Scalar Fields}
\par In order to approach inflationary perturbations and any non-Gaussianity arising from them, it is advantageous to adopt the so-called $\delta N$ formalism used in \cite{curvaton} and \cite{modul}. This formalism is used to quantify the number of $e$-folds that one region is ahead of another (due to perturbations), where essentially, $\delta N$ is analogous to the curvature perturbation (on some uniform-energy-density hypersurface) $\zeta$, and is advantageous when used at sufficiently large scales where any spatial gradient can be neglected. Specifically, $\zeta$ can be defined at some time $t_f$ as
\begin{equation}\label{zeta}
    \zeta(t_f,\vec{x})=N(t_f,t_i,\vec{x}) - \left<N(t_f,t_i,\vec{x})\right>,
\end{equation}
where $N(t_f,t_i,\vec{x})\equiv\displaystyle\int_{t_*}^{t_f}H(t,\vec{x})\diff t$ is the number of $e$-folds, with $t_i$ some initial time, and the angular brackets denote a spatial average (such that the second term in \eqref{zeta} is analagous the the expansion of the unperturbed ``background'' spacetime) \cite{curvaton}\cite{modul}. Letting $t_i=t_*$ be the time of horizon crossing, $\zeta$ can expanded written in terms of the value(s) of the scalar field(s) (whichever field that may be) at $t_*$,
\begin{equation}\label{zap}
    \zeta(t_f)=N_a\delta\phi^a_* + \frac{1}{2}N_{ab}\delta\phi^a_*\delta\phi^b_* + \mathcal{O}(\delta\phi_*^3),
\end{equation}
where $\delta\phi_*^a$ is the perturbation of the scalar field $\phi^a$ at $t_*$, $\delta\phi_*^a$ is defined similarly with some $\phi^b$ (here, I only consider the presence of at most two scalar fields), and
\begin{equation}
N_a\equiv\pdv{N}{\phi^a}, \quad N_{ab}\equiv\pdv[2]{N}{\phi^a}{\phi^b}\quad\text{\cite{curvaton}}.
\end{equation}
Now, the primordial power spectrum of $\zeta$ is defined as
\begin{equation} \label{PwrDef}
    \left<\zeta_{\vec{k}_1}\zeta_{\vec{k}_2}\right>\equiv(2\pi)^3P_\zeta(k_1)\delta(\vec{k}_1+\vec{k}_2),
\end{equation}
and taking the first order approximation of \eqref{zap} (with two scalar fields indexed by $a$ and $b$ with perturbation wavenumbers $\vec{k}_1$ and $\vec{k}_2$, respectively), I find that
\begin{equation}\label{zetas}
    \left<\zeta_{\vec{k}_1}\zeta_{\vec{k}_2}\right>=\left<N_a\delta\phi^a_*N_b\delta\phi^b_*\right>=N_aN_b\left<\delta\phi^a_*\delta\phi^b_*\right>.
\end{equation}
The quantity $\left<\delta\phi^a_*\delta\phi^b_*\right>$ can be expressed in terms of the standard power spectrum of any two scalar fields $P^{ab}(k)$, 
\begin{equation}\label{expphi}
\left<\delta\phi^a_*\delta\phi^b_*\right>=(2\pi)^3P^{ab}(k_1)\delta(\vec{k}_1+\vec{k}_2),
\end{equation}
and assuming that the two fields are uncorrelated and have the same amplitude, \eqref{expphi} will only be nonzero for $a=b$, in which case it will follow the standard single light scalar field power spectrum $P(k)$, i.e.
\begin{equation}\label{Pab}
    P^{ab}(k)=\delta^{ab}P(k)=\delta^{ab}\frac{2\pi^2}{k^3}\left(\frac{H_*}{2\pi}\right)^2,
\end{equation}
where $H_*\equiv H(t_*)$. It follows that, plugging \eqref{Pab} into \eqref{expphi} and that into \eqref{zetas},
\begin{equation}
     \left<\zeta_{\vec{k}_1}\zeta_{\vec{k}_2}\right>=N_aN_b(2\pi)^3\delta^{ab}\frac{2\pi^2}{k^3}\left(\frac{H_*}{2\pi}\right)^2\delta(\vec{k}_1+\vec{k}_2)=(2\pi)^3P_\zeta(k_1)\delta(\vec{k}_1+\vec{k}_2),
\end{equation}
so that, following the $\delta^{ab}$ and defining the dimension\textit{less} power spectrum $\mathcal{P}_\zeta(k)=\frac{k^3}{2\pi^2}P_\zeta(k)$,
\begin{equation}\label{Presult}
    P_\zeta(k)=N_a^2\frac{2\pi^2}{k^3}\left(\frac{H_*}{2\pi}\right)^2 \quad \Rightarrow \quad \boxed{\mathcal{P}_\zeta(k)=\left(N_a\frac{H_*}{2\pi}\right)^2}.
\end{equation}
In the following sections, I will determine the form of $N_a$ for the curvaton and modulating reheating scenarios, and show how that will determine the power spectrum and the observable parameters derived from it.
\section{The Curvaton}\label{curvatonsection}
\par Imagine that during inflation, there is additional scalar field $\sigma_c$ known as the \textit{curvaton} (named after the curvature perturbations it produces) present alongside the inflaton, that is subdominant in mass relative to the curvaton. Here, I consider a curvaton with a quadratic potential $U(\sigma)=\frac{1}{2}m_{\sigma_c}^2\sigma_c^2$, where $m_{\sigma_c}$ is the mass of the curvaton, following \cite{curvaton}. For the purposes of this paper, I will summarize much of the following intermediate steps that lead to the important results. In order to examine the effect of a curvaton on $\mathcal{P}_\zeta$ and related parameters, it's necessary to consider $N_R$, which defined as the number of $e$-folds from some time after the decay of the inflaton, to some time well after the curvaton has decayed \cite{curvaton}. The reason that $N_R$ is of interest is that the curvaton only produces noticeable perturbations once the inflaton has decayed and is no longer the ``primary'' scalar field at play. After analyzing the properties of the curvaton over the course of its evolution, \cite{curvaton} arrives at the quantity $Q$, which schematically represents the part of $N_R$ which depends on the curvaton at horizon entry, $\sigma_{c,*}$, and is also a function of the decay rate of the curvaton $\Gamma_{\sigma_c}$, and $m_{\sigma_c}$. As such, going back to \eqref{Presult}, it follows that $N_a=N_{\sigma_c}=Q_c$ for the curvaton, where $Q_{\sigma_c}\equiv\pdv{Q}{\sigma_c}$. For the inflaton with some potential $V(\phi)$, we have that $N_a=N_\phi\equiv \frac{1}{M_\text{pl}^2}\frac{V}{V_\phi}$, where $V_\phi\equiv\pdv{V}{\phi}$, which simply comes from the standard calculation of solo inflaton $e$-folds  \cite{curvaton}. Plugging these different $N_a$ terms into \eqref{zap} to obtain a $\zeta$ (to first order) for each field, the same process of \eqref{PwrDef} - \eqref{Presult} can be applied (after adding the contribution of each field to the overall power spectrum) to obtain 
\begin{equation}\label{Pzc}
    \mathcal{P}_{\zeta,c}=\left(\frac{1}{M_\text{pl}^4}\frac{V^2}{V^2_\phi}+Q_{\sigma_c}^2\right)\left(\frac{H_*}{2\pi}\right)^2.
\end{equation}
\par From \eqref{Pzc}, the spectral index $n_s$, its running $n_\text{run}$, and the tensor-to-scalar ratio $r$ can finally be obtained, which are parameters that can be constrained by observations.
By definition, using $k=aH_*$,
\begin{equation}\label{ns}
    n_s-1\equiv\frac{\diff \log \pc}{\diff\log k}=\frac{k}{\pc}\dv{\pc}{k}=\frac{k}{\pc}\left\{\frac{1}{M_\text{pl}^4}\dv{}{k}\left[\frac{V^2}{V^2_\phi}\left(\frac{H_*}{2\pi}\right)^2\right]+\dv{}{k}\left[Q_{\sigma_c}^2\left(\frac{H_*}{2\pi}\right)^2\right]\right\}.
\end{equation}
This is further simplified using chain-rule relations such as $\dv{V}{k}=\dv{\phi}{k}V_\phi$, $\dv{V_\phi}{k}\equiv\dv{\phi}{k}V_{\phi\phi}$, and $\dv{H_*}{k}=\frac{1}{a}$. After doing the algebra and defining the first-order slow-roll parameters (SRP) as
\begin{equation}
    \epsilon_V\equiv\frac{1}{2}M^2_\text{pl}\left(\frac{V_\phi}{V}\right)^2\qquad \eta_V\equiv M^2_\text{pl}\frac{V_{\phi\phi}}{V},
\end{equation}
an expression for the spectral index can be obtained \cite{curvaton} (to first order in SRP) in terms of the SRP, 
\begin{equation}
    n_s-1=-2\epsilon_V-\frac{4\epsilon_V-2\eta_V}{1+2\epsilon_VM^2_\text{pl}Q_{\sigma_c}^2}.
\end{equation}
Similarly, following the same steps, the running of the spectral index can also be obtained as
\begin{equation}
    n_\text{run}=\dv{n_s}{\log k}=k\dv{n_s}{k},
\end{equation}
where the SRP parameters within $n_s$ can be differentiated with respect to $k$ using their definitions and the chain-rule relations used in \eqref{ns}, to obtain
\begin{equation}
    n_\text{run}=-4\epsilon_V(2\epsilon_V-\eta_V)-\frac{2\left[8\epsilon_V^2-6\epsilon_V\eta_V+2\epsilon_V(2\epsilon_V\eta_V-2\eta_V^2)M^2_\text{pl}Q_{\sigma_c}^2\right]}{\left(1+2\epsilon_VM^2_\text{pl}Q_{\sigma_c}^2\right)^2}\quad\text{\cite{curvaton}}.
\end{equation}
Now, to find $r$, note that during inflation, \textit{tensor} perturbation modes are generated following a power spectrum of
\begin{equation}
    \mathcal{P}_T=\frac{8}{M^2_\text{pl}}\left(\frac{H_*}{2\pi}\right)^2 \quad \text{\cite{curvaton}}.
\end{equation}
The tensor to scalar ratio is defined to be the ratio between power spectra of tensor and scalar perturbations, so I find that (using the definition for $\epsilon_V$)
\begin{equation}
    r\equiv\frac{\mathcal{P}_T}{\pc}=\frac{\frac{8}{M^2_\text{pl}}\left(\frac{H_*}{2\pi}\right)^2}{\left(\frac{1}{M_\text{pl}^4}\frac{V^2}{V^2_\phi}+Q_{\sigma_c}^2\right)\left(\frac{H_*}{2\pi}\right)^2}=\frac{\frac{8}{M^2_\text{pl}}}{\frac{1}{2M_\text{pl}^2\epsilon_V}+Q_{\sigma_c}^2}=\frac{16\epsilon_V}{1+2\epsilon_VM^2_\text{pl}Q_{\sigma_c}^2}.
\end{equation}
\section{The Modulated Reheating Scenario}
\par The second spectator field scenario that I will consider is known as the \textit{modulated reheating scenario} \cite{modul}. In traditional inflation theory, the inflaton decays uniformly during reheating; the crux of this theory is that the inflaton can have a spatially modulating decay rate if an additional so-called \textit{modulus} spectator field, denoted with $\sigma_{m}$, is present (with an energy density much lower than that of the inflaton).
\par To begin, I'll use the same $\delta N$ formalism presented in equations \eqref{zeta} - \eqref{Presult} (which can still be applied because I did not specify the spectator field being the curvaton until section \ref{curvatonsection}), and apply \eqref{Presult} to $\sigma_m$ to obtain
\begin{equation}
    \pmo =\left(\pdv{N}{\sigma_m}\frac{H_*}{2\pi}\right)^2,
\end{equation}
where again, I assume that the modulus field is light, $N$ denotes the number of $e$-folds, and $H_*$ is the value of $H$ upon horizon entry. To proceed, I will assume that the Universe transitions from inflaton oscillation to radiation dominated instantaneously, i.e. it enters reheating instantly, following \cite{modul}.
\par I will next assume that $\sigma_\text{reh}=\sigma_*$, i.e. that the modulus field does not change between horizon entry and the beginning of reheating, following \cite{modul}.
\section{Chaotic Inflation}
\par In \cite{curvaton}, one of the primary inflaton models discussed is \textit{Chaotic inflation}, which is characterized by the potential
\begin{equation}
    V=\lambda M^4_\text{pl}\left(\frac{\phi}{M_\text{pl}}\right)^n,
\end{equation}
where $\lambda$ is a parameter fixed by observations. Here, I will discuss the case of $n=4$, which is the primary focus of \cite{curvaton}, given that other values such as $n=2,6$ are not as affected by the inclusion of a curvaton \cite{curvaton}.
Consider $\sigma_* \ll M_\text{pl}$ and $p\ll 1$

\section{Observational Consequences and Discussion}
\par As shown in \cite{duration}, multiple field models can provide strong fits to recent observational data.
\par Shown in \cite{duration}, different multiple field models can be distinguished
\cite{curvaton}
\cite{modul}
\cite{trispec}
\cite{duration}
\cite{waterfall}
\section*{References}
\printbibliography[heading=none]
\begin{equation}
    %\LARGE{K^{A^{L^{L^{E^{E^{N}}}}}}   = raider^2}
\end{equation}

\end{document}